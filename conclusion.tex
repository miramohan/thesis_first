\chapter{CONCLUSION}
\label{chap:conclusion}
A new dual-sim or sim after sim flow for gatesim was developed. Simulation performance when benchmaked and compared against current co-simulation method on a dedicated machine, shows a consistent improvement of around 10. Use of FSDB as the source of test vectors helps to keep the disk requrirement minimal and to get rid of bulky time hogging test bench components. The methodology also enables quicker turn-around time. This 10 times improvement in simulation performance will help in aiding time-to-market of cutting-edge processors. The method still requires a one time RTL simulation runs for generating stimulus (as FSDB). The proposed methodology does only minor modifications to existing co-simulation methodology and hence could be adopted to new projects with ease.
\paragraph{Potential for future work:}Observed performance of improved dual-sim methodology could be further increased by optimizing glue code used for reading FSDB and applying that inside design. Infrastructure created during the course of this project could be packaged into libraries to simplify porting between projects.
