\section*{\centering ABSTRACT}
\newcommand{\RNum}[1]{\uppercase\expandafter{\romannumeral #1\relax}}


\centerline{\emph{\bf PART \RNum{1}- Interactive Offline Interface To Debug Simulation Failure }}
\vspace{5pt}
Processor execution logs created during simulation, contains in depth details pertaining to processor execution.  In the event of a simulation failure, debugging necessitates tracing through the execution logs for failure diagnosis.  Due to comprehensive information contained in these log files, it becomes overwhelming to comprehend it quickly enough as information is spread across. Such manual tracing is error-prone and time consuming.
\newline
This project aims to implement a GUI \nomenclature{GUI}{Graphical User Interface} and there by enable effortless, faster execution debug even from remote locations. The interface gathers data from multiple sources related to execution flow and represents it correlated, as appropriate. The graphic interactive navigation windows attempt to reduce the user's time spends on tracing cause of failure. The project attempts to use web based technologies to enable remote debug.



\paragraph{Keywords:}
 \emph{Debug Interface, Execution Log Files, SoC  Verification.}


 \paragraph{}

\centerline{\emph{\bf PART \RNum{2}- Improved Dual Simulation Approach To Gatesim}}
\vspace{5pt}
Gatesim or gate level simulation verification focuses on verifying the post layout netlist of the design. Gatesim verification is an important milestone and confidence builder for verification. Multiple methodologies are in existence in the industry for this purpose. In AMD, Gatesim verification uses co-simulation methodology for the purpose, where full chip behavioral RTL \nomenclature{RTL}{Register Transfer Level description of circuit} and gate netlist are simulated simultaneously in one simulation. Verification is achieved by driving corresponding RTL stimulus onto netlist and by comparing response every cycle.
\newline
The objective of this project is to improve the simulation turn-around times and reducing resource requirements involved in Gatesim verification without compromising on verification. The thesis proposes a new dual simulation approach to Gatesim where RTL and gate level simulations are performed separately, by exporting test vectors for netlist simulation from RTL simulation. The approach attempts to overcome performance issues with current co-simulation methodology.

\paragraph{Keywords:}
 \emph{Gatesim,  GLS,  Co-simulation Methodology,  Functional Verification, Netlist Simulation.}


