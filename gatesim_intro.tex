\chapter{INTRODUCTION}
%\section{Background}
%\section{Purpose/Motivation}
%\section{Approach}
%\section{Main Contributions}
%\section{Organization of thesis}

Functional verification of a design is the task of verifying that the logic design conforms to specification and ensures functional correctness of the design.  With the rapid increase in design complexity and size, this phase of circuit design has become the most crucial and resource-intensive. It is widely accepted that more than 70 percent of design efforts is spend on verification and with the advancement in silicon technology and adoption of complex SoC designs; this situation is only projected to get worse in future. 

Verification is done at different abstraction levels. Two main abstraction levels from verification point of view are RTL and gate level. RTL enables relatively easier management of complex designs and has less verification time requirements compared to gate or netlist level. However it cannot eliminate the need for verification at netlist level as each level is required for specific type of verification. While RTL verification is apt for functional validation or architectural analysis, more detailed analysis like timing, power etc require detailed models of lower levels. Another important challenge is ensuring the functional equality of models at different abstraction level.  

 Traditionally simulation based verification has be used as the primary approach for verification at both RTL as well as gate level.  But with very complex designs, this approach has become inefficient in finding subtle design bugs. Also at gate level, simulation based verification takes rather too much time that an exhaustive verification is impossible. On the other hand, formal verification tools have gained popularity since it can mathematically prove or disprove the design validity. These mathematical methods require less manual effort than simulation based verification and hence a lot faster. However these mathematical models cannot comprehend all kind of complexities that could occur in the design and it is very much clear that this alone can solve all issues. Rather a mix of simulation and formal verification methods are practiced to ensure all corner cases are covered. 


\section{GATE LEVEL SIMULATION}

Gate Level Simulation (also called as gatesims) still play crucial role in verification, in spite of advancement in formal verification techniques like LEC \nomenclature{LEC}{Logic Equivalence Check} and STA \nomenclature{STA}{Static Timing Analysis}.   When having to verify with gatesim one has to start planning early, as it has to pass through various stages before sign off. The setup for gate level simulation starts right after the prelim netlist is available and will continue till final post layout netlist.  

Even though gatesims help in finding many issues related to timing and power, it is considered as a ``{\it necessary evil}'' by engineers. This is mainly because gate level simulation is inherently very slow. It is also hard to find the optimal list of test cases to effectively utilize gatesims. Debugging is also very tedious process at gate level combined with the long run time makes turn-around for debug long hence ultimately affects time-to-market of the product.

Various approaches have been adopted over the years for gate level simulation with each method trying to improve simulation performance and ease of debug. Since generally netlists are Verilog based, test environments used for RTL verification could be reused for netlist verification by replacing RTL with netlist as appropriate. It should also be possible to have the test vectors generated for RTL simulation used as stimulus for netlist simulation. Such application of testvectors can be done either by having the RTL be simulated in parallel with gatesim or by applying captured the test vectors from RTL simulation onto netlist simulation. The first method of stimulus application is called a co-simulation approach and the second is called a dual-simulation or sim after sim approach. Each method has its own set of pros and cons. Simulation time, design complexity, memory, ease of debug, testbench complexity tradeoffs are considered while choosing any particular method for gatesims in a project. This thesis analyzes the advantage and disadvantage of past and present gatesim methodologies and  proposes a new improved sim after sim or dual simulation approach to gatesims. 



\section{ORGANIZATION OF THE THESIS}
The organization of this project report is as follows:\\
\noindent 
{\bf Chapter}~\ref{chap:gate_intro.tex} -{\it Gate Level Simulation} explains the relevance, advantages and limitations of gate level simulations.\\
{\bf Chapter}~\ref{chap:methodologies.tex} -{\it Gatesim Methodologies} briefly explains various approaches used for gatesims, their advantages and disadvantages.\\
{\bf Chapter}~\ref{chap:dualsim.tex} -{\it Improved Dual-Sim Approach To Gatesim} describes the implementation and flow of proposed dual sim approach .\\
{\bf Chapter}~\ref{chap:results.tex} -{\it Results} gives comparison of simulation performances and memory utilization of current and proposed approaches.\\
%{\bf Chapter}~\ref{chap:GUI_results.tex} show the GUI windows and    .\\
%{\bf Chapter}~\ref{chap:conclusion} discusses the various 
%conclusion drawn from the results and the scope for future work.

